\chapter{Introduction}
\label{chapter:intro}

% Use full justification for chapter text
\justifying

\section{Background to the Study}

Being able to grow in one's career is an important aspect that individuals consider when making choices such as furthering their education. Some weigh the pros and cons of investing in their studies \textcite{david_ba_eres_99a970a4}. Investments in further education or professional training often involve significant financial cost and uncertainty regarding future returns \textcite{stephen_c__poplaski_4baca635}. For companies, it is a similar problem. While using an individualistic approach to training may improve employee productivity and increase retention, it also has the risk that employees may leave after they have upskilled themselves \textcite{angela_sutan_88b257bc}. On the other hand, minimal investment in development may result in many skill gaps and reduced organisational performance \textcite{esmond_naalu_7803d844}.

For a company looking to invest in their employees, an individual development plan is vital for their development. They would then have access to tailored development plans \textcite{besim_mustafa_0600d17d}. IDPs are intended to identify skill gaps, guide training interventions, and support career progression in a structured manner \textcite{deborah_e__eason_1beca78d}. When this is done effectively, training resources would be allocated effectively, making sure training is aligned with individual needs rather than applying them in a generic manner. Determining the right development actions for an individual is a complex task that often rely on scattered historical data and stale job descriptions \textcite{chao_wang_f11198d9}. In such scenarios, capturing the true individual needs would prove to be difficult.

Telecom Namibia is a local telecommunications company that has this process outlined as follows. The line manager drafts a Job Competency Profile (JCP), this profile contains all the information on what it means to be competent at a specific job. Then the employee rates themselves and the supervisor does a rating as well, creating the Employee Competency Profile (ECP). This describes the employee's delta of their job competency. The supervisor then discusses the way forward, highlighting what training courses to take, development actions, career transitioning, and possible mentorship opportunities. Thus, drafting the Individual Development Plan.  Although this is a well-designed process that follows a systemic framework, it relies heavily on manual analysis and judgment. This leads to scaling, maintainability, and consistency issues.

This process has been digitalized by the creation of TrainEase and the SkillHarbor platform, up until the point of generating an ECP.  However, the subsequent stages of IDP creation remains a manual process. We now look at how can we make use of deep learning techniques to aid the process of IDP creation. This paper documents the process of how we aim to implement a recommender system that would help employees/employers make informed decisions on their development plans. These recommendations would not be set in stone, but the users would have the option to approve or disprove of the recommendation and further improve future recommendations.

This study delves into the use cases of deep learning recommender systems to support the creation of individual development plans. Specifically, it examines how machine learning models can assist in recommending training courses, development actions, career transition possibilities, and mentorship opportunities. By building and integrating these models into the existing platform, we aim to demonstrate how the machine learning can enhance the personalisation of these development plans.
\section{Choice of Methods}
The study adopts a mixed-methods approach, integrating qualitative insights and quantitative validation \textcite{oliveira2021mixed, guetterman2016distinguishes}. Data collection will involve semi-structured interviews and structured questionnaires. Analysis will include thematic coding and statistical modeling, supporting deep learning model training and evaluation.



\section{The Research Time Horizon}
A longitudinal design will be employed to monitor participants' development plans over time. Regular assessments will capture evolving employee competencies and measure the recommender system's long-term effectiveness \parencite{kelley2011sample, jung2023longitudinal}. Careful sampling and structured engagement strategies will maintain data integrity and participant involvement.

\section{Problem Statement}
Telecom Namibia currently conducts competency assessments manually using Excel templates and email exchanges. This process is inefficient, prone to errors, and lacks real-time insights, thereby hindering timely decision-making. As the organization grows, scalability issues become apparent, necessitating an automated, centralized platform for more efficient and accurate IDP creation.

\begin{figure}[H]
    \centering
    \begin{tikzpicture}[scale=0.9, transform shape]

        % Supervisor side (left)
        \node[shaded-primary] (supCreate) {Supervisor drafts JCP};
        \node[shaded-primary, below=of supCreate] (supFinalize) {Supervisor finalises ECP};

        % Employee side (right)
        \node[block-primary, right=4cm of supCreate] (empReceive) {Employee receives template};
        \node[block-primary, below=of empReceive] (empFill) {Employee completes template};

        % Senior Manager (center bottom)
        \node[block-primary, below=1cm of supFinalize, text width=5cm, align=center] (seniorManager) {Senior Manager evaluates ECP and recommends Development Plan};

        % Horizontal arrows (Supervisor to Employee)
        \draw[line] (supCreate.east) -- node[midway, above] {Send Template} (empReceive.west);
        \draw[line] (empFill.west) -- node[midway, below] {Return Completed} (supFinalize.east);

        % Vertical arrows
        \draw[line] (empReceive) -- (empFill);

        % Final vertical arrow to Senior Manager
        \draw[line] (supFinalize) -- (seniorManager);

    \end{tikzpicture}
    \caption{Current IDP Creation Process at Telecom Namibia (Interaction View).}
    \label{fig:idp_process_interaction}
\end{figure}

\section{Research Aim}
The aim is to enable access to real-time data,through a platform where management, staff and supervisors may interact through training and skill audit assessments. This would automate workflows associated with individual development plan formulation and implementation.



\section{Research Questions}
\textbf{In what ways may deep learning techniques be applied to create a personalized and efficient recommender model that appropriately evaluates employee competencies and suggests chances for customized growth for individual development plans?}
\ResearchQuestion{1}{Which employee data types, e.g., skills, evaluations, performance reviews, etc, are most crucial to the construction of a well performing deep learning model that generates recommendations for individual development plans?}

\ResearchQuestion{2}{How can employee characteristics and past performance data be used to refine deep learning algorithms to produce precise and customized recommendations for development plans?}

\ResearchQuestion{3}{In what ways can the use of real-time performance data and feedback enhance the individual growth plan recommendations' adaptability and relevance?}

\ResearchQuestion{4}{What is the difference between the accuracy, employee skill development, and organizational productivity of the deep learning-based recommender system for individual development plans and the conventional approaches?}
\section{Research Objectives}

\ResearchObjective{The primary objective of this research is to design an individual development plan recommender model using deep learning.}

\begin{itemize}
    \item \textbf{RO1:} To compile comprehensive datasets containing employee skills, evaluations, and development history.
    \item \textbf{RO2:} To design a deep learning model that analyzes skill gaps and generates customized growth plans.
    \item \textbf{RO3:} To integrate real-time feedback loops to dynamically update employee development recommendations.
    \item \textbf{RO4:} To evaluate the effectiveness of the proposed system compared to traditional IDP development methods.
\end{itemize}
\section{Rationale and Motivation}
The rapid evolution of employee competencies and organizational objectives makes manual development planning increasingly unsustainable. Automating IDP generation through AI enables scalability, accuracy, and responsiveness. Personally, this project reflects my passion for applying machine learning to real-world human resource development challenges, specifically optimizing organizational talent management systems.

\section{Research Approach Overview}
The research follows an interpretivist philosophical stance, utilizing an abductive mixed-methods strategy. Data will be collected from surveys, interviews, and organizational records. Analyses will include both thematic qualitative coding and quantitative statistical techniques, supporting the development of a deep learning recommender system.





\subsection{Interpretivism as a Research Philosophy}
\begin{sloppypar}
Interpretivism emphasizes the individuality of human experiences \parencite{irshaidat2019interpretivism, Saunders2012}. Each employee's development needs are influenced by personal, social, and cultural factors \parencite{myers2008qualitative}. Interpretivism opposes generalization and instead values subjective interpretations, aligning well with the need for personalized development plans.
\end{sloppypar}

\subsection{The Research Approach}
Following \textcite{mantere2013reasoning, hurley2021integrating}, an abductive approach combines inductive and deductive reasoning. \textcite{thompson2022guide} provides an 8-step abductive framework, guiding data collection, feature extraction, theme development, theorizing, model implementation, and evaluation through continuous reflection.

\subsection{Research Strategy: Action Research}
\begin{sloppypar}
Action Research (AR), particularly Canonical Action Research (CAR) as outlined by \textcite{avison1999action, davison2021research}, is adopted. AR fosters collaboration between researcher and participants in an iterative cycle of problem identification, intervention, observation, and reflection---making it highly suitable for information system development like the proposed IDP recommender.
\end{sloppypar}

\section{Expected Deliverables and Contributions}
\begin{itemize}
    \item A fully functional deep learning-based IDP recommender model.
    \item A prototype centralized platform for employee development planning.
    \item Contributions to the field of AI-driven HR systems and personalized learning pathways.
\end{itemize}

\section{Structure of the Thesis}
\begin{itemize}
    \item \textbf{Chapter 1: Introduction} � Context, problem statement, research questions, objectives, and thesis outline.
    \item \textbf{Chapter 2: Literature Review} � Systematic review of IDPs, recommender systems, deep learning techniques, and related work.
    \item \textbf{Chapter 3: Theoretical Framework} � Mathematical foundations of recommendation systems, deep learning architectures, and evaluation metrics.
    \item \textbf{Chapter 4: Research Design} � Research methodology, model selection rationale, and experimental design.
    \item \textbf{Chapter 5: Implementation} � System architecture, data pipeline, model implementations, and platform integration.
    \item \textbf{Chapter 6: Results and Evaluation} � Experimental results, model comparisons, and performance analysis.
    \item \textbf{Chapter 7: Conclusion} � Summary of findings, research contributions, and limitations.
    \item \textbf{Chapter 8: Recommendations} � Future research directions and practical recommendations.
\end{itemize}
